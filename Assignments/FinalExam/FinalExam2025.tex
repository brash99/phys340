\documentclass{article}
\usepackage{amsmath}
\usepackage{amssymb}
\usepackage{enumerate}

\begin{document}

\title{Physics 340 Final Exam: Mathematical Physics}
\author{}
\date{}
\maketitle

\section*{Instructions}
Answer all five questions. Show all work for full credit. Each question is worth 20 points.

\section*{Question 1: Algebra of Vectors (20 points)}

Consider the following two vectors in \( \mathbb{R}^3 \):
\[
\mathbf{A} = 3\hat{i} - 2\hat{j} + \hat{k}, \quad \mathbf{B} = \hat{i} + 4\hat{j} - 2\hat{k}.
\]
\begin{enumerate}[a)]
  \item Compute the dot product \( \mathbf{A} \cdot \mathbf{B} \).
  \item Compute the cross product \( \mathbf{A} \times \mathbf{B} \).
  \item Find the angle between the two vectors \( \mathbf{A} \) and \( \mathbf{B} \).
\end{enumerate}

\newpage
Continue your solution here
\vspace{4.0in}

\newpage

\section*{Question 2: Coupled Linear First-Order Differential Equations (20 points)}

Consider the coupled system of differential equations:
\[
\frac{dx}{dt} = 4x + y, \quad \frac{dy}{dt} = -2x + y.
\]
\begin{enumerate}[a)]
  \item Write the system in matrix form: \( \frac{d\vec{X}}{dt} = A \vec{X} \).
  \item Find the eigenvalues and eigenvectors of matrix \( A \).
  \item Solve the system for \( x(t) \) and \( y(t) \), assuming initial conditions \( x(0) = 2 \), \( y(0) = 1 \).
  \item Describe, in words, the behavior of this system.
\end{enumerate}

\newpage
Continue your solution here
\vspace{4.0in}

\newpage

\newpage
Continue your solution here
\vspace{4.0in}

\newpage

\section*{Question 3: AC RL Circuit and Complex Impedance (15 points)}

An alternating voltage source \( V(t) = V_0 e^{i\omega t} \) is applied across a resistor \( R \) and an inductor \( L \) connected in series.

\begin{enumerate}[a)]
  \item (5 points) Using the concept of complex impedance, derive the total impedance \( Z \) of the RL circuit in terms of \( R \), \( L \), and the angular frequency \( \omega \).
  \item (5 points) Using Ohm's law in the complex form \( V(t) = I(t) Z \), find the expression for the current \( I(t) \) in the circuit. Express the current in terms of \( V_0 \), \( \omega \), \( R \), and \( L \).
  \item (5 points) Find the phase shift between the voltage and the current in the circuit. What does this phase shift physically represent in terms of the behavior of the resistor and the inductor?
\end{enumerate}

\newpage
Continue your solution here
\vspace{4.0in}

\newpage

\newpage
Continue your solution here
\vspace{4.0in}

\newpage

\section*{Question 4: Fourier Analysis (20 points)}

Let \( f(x) \) be a periodic function with period \( T \), defined as:
\[
f(x) = \begin{cases}
  1, & 0 \leq x < \frac{T}{2}, \\
  -1, & \frac{T}{2} \leq x < T.
\end{cases}
\]
\begin{enumerate}[a)]
  \item Compute the Fourier coefficients \( a_0 \), \( a_n \), and \( b_n \) for this function.
  \item Find the Fourier series representation of \( f(x) \).
  \item Discuss the convergence of the Fourier series for this function.
\end{enumerate}

\newpage
Continue your solution here
\vspace{4.0in}

\newpage

\newpage
Continue your solution here
\vspace{4.0in}

\newpage

\section*{Question 5: Vector Calculus (20 points)}

Let \( \mathbf{F} = x^2 \hat{i} + 2xy \hat{j} + yz \hat{k} \) be a vector field.

\begin{enumerate}[a)]
  \item Compute the divergence of \( \mathbf{F} \), \( \nabla \cdot \mathbf{F} \).
  \item Compute the curl of \( \mathbf{F} \), \( \nabla \times \mathbf{F} \).
  \item Evaluate the line integral of \( \mathbf{F} \) along the curve \( C \) parameterized by \( \mathbf{r}(t) = \langle t, t^2, t^3 \rangle \) for \( t \) from 0 to 1.
\end{enumerate}

\newpage
Continue your solution here
\vspace{4.0in}

\newpage

\newpage
Continue your solution here
\vspace{4.0in}

\newpage

\end{document}


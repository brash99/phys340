\documentclass{article}
\usepackage{amsmath}
\usepackage{amssymb}
\usepackage{enumerate}
\usepackage{graphicx}

\begin{document}

\title{Solutions: Mathematical Physics Final Exam}
\author{}
\date{}
\maketitle

\section*{Question 1: Algebra of Vectors}

Consider the following two vectors in \( \mathbb{R}^3 \):
\[
\mathbf{A} = 3\hat{i} - 2\hat{j} + \hat{k}, \quad \mathbf{B} = \hat{i} + 4\hat{j} - 2\hat{k}.
\]

\begin{enumerate}[a)]
  \item Compute the dot product \( \mathbf{A} \cdot \mathbf{B} \).
  
  \[
  \mathbf{A} \cdot \mathbf{B} = 3 \times 1 + (-2) \times 4 + 1 \times (-2) = 3 - 8 - 2 = -7.
  \]

  \item Compute the cross product \( \mathbf{A} \times \mathbf{B} \).
  
  \[
  \mathbf{A} \times \mathbf{B} = \begin{vmatrix} \hat{i} & \hat{j} & \hat{k} \\ 3 & -2 & 1 \\ 1 & 4 & -2 \end{vmatrix}
  = \hat{i} \begin{vmatrix} -2 & 1 \\ 4 & -2 \end{vmatrix} - \hat{j} \begin{vmatrix} 3 & 1 \\ 1 & -2 \end{vmatrix} + \hat{k} \begin{vmatrix} 3 & -2 \\ 1 & 4 \end{vmatrix}.
  \]
  
  \[
  = \hat{i} \left[ (-2)(-2) - (1)(4) \right] - \hat{j} \left[ (3)(-2) - (1)(1) \right] + \hat{k} \left[ (3)(4) - (-2)(1) \right]
  \]
  \[
  = \hat{i} \left[ 4 - 4 \right] - \hat{j} \left[ -6 - 1 \right] + \hat{k} \left[ 12 + 2 \right]
  \]
  \[
  = 0\hat{i} + 7\hat{j} + 14\hat{k}.
  \]
  Therefore, 
  \[
  \mathbf{A} \times \mathbf{B} = 7\hat{j} + 14\hat{k}.
  \]

  \item Find the angle between the two vectors \( \mathbf{A} \) and \( \mathbf{B} \).

  The angle \( \theta \) between the vectors is given by:
  \[
  \cos \theta = \frac{\mathbf{A} \cdot \mathbf{B}}{|\mathbf{A}| |\mathbf{B}|}
  \]
  First, compute the magnitudes of \( \mathbf{A} \) and \( \mathbf{B} \):
  \[
  |\mathbf{A}| = \sqrt{3^2 + (-2)^2 + 1^2} = \sqrt{9 + 4 + 1} = \sqrt{14}, \quad |\mathbf{B}| = \sqrt{1^2 + 4^2 + (-2)^2} = \sqrt{1 + 16 + 4} = \sqrt{21}.
  \]
  Now, compute the cosine of the angle:
  \[
  \cos \theta = \frac{-7}{\sqrt{14} \times \sqrt{21}} = \frac{-7}{\sqrt{294}}.
  \]
  Hence,
  \[
  \theta = \cos^{-1} \left( \frac{-7}{\sqrt{294}} \right).
  \]

\end{enumerate}

\section*{Question 2: Coupled Linear First-Order Differential Equations}

Consider the coupled system of differential equations:
\[
\frac{dx}{dt} = 4x + y, \quad \frac{dy}{dt} = -2x + y.
\]

\begin{enumerate}[a)]
  \item Write the system in matrix form: \( \frac{d\vec{X}}{dt} = A \vec{X} \).
  
  \[
  \vec{X} = \begin{bmatrix} x \\ y \end{bmatrix}, \quad A = \begin{bmatrix} 4 & 1 \\ -2 & 1 \end{bmatrix}, \quad \frac{d\vec{X}}{dt} = A \vec{X}
  \]

  \item Find the eigenvalues and eigenvectors of matrix \( A \).

  The eigenvalues are the solutions to \( \det(A - \lambda I) = 0 \):
  \[
  \det \begin{bmatrix} 4 - \lambda & 1 \\ -2 & 1 - \lambda \end{bmatrix} = (4 - \lambda)(1 - \lambda) + 2 = \lambda^2 - 5\lambda + 6 = 0.
  \]
  Solving this quadratic equation gives the eigenvalues:
  \[
  \lambda = 2, 3.
  \]
  Now, find the eigenvectors:
  - For \( \lambda = 2 \), solve \( (A - 2I)\vec{v} = 0 \):
  \[
  \begin{bmatrix} 2 & 1 \\ -2 & -1 \end{bmatrix} \begin{bmatrix} v_1 \\ v_2 \end{bmatrix} = \begin{bmatrix} 0 \\ 0 \end{bmatrix}.
  \]
  This gives the eigenvector \( \vec{v_1} = \begin{bmatrix} 1 \\ -2 \end{bmatrix} \).
  
  - For \( \lambda = 3 \), solve \( (A - 3I)\vec{v} = 0 \):
  \[
  \begin{bmatrix} 1 & 1 \\ -2 & -2 \end{bmatrix} \begin{bmatrix} v_1 \\ v_2 \end{bmatrix} = \begin{bmatrix} 0 \\ 0 \end{bmatrix}.
  \]
  This gives the eigenvector \( \vec{v_2} = \begin{bmatrix} 1 \\ -1 \end{bmatrix} \).

  \item Solve the system for \( x(t) \) and \( y(t) \), assuming initial conditions \( x(0) = 2 \), \( y(0) = 1 \).

  The general solution is:
  \[
  \vec{X}(t) = c_1 e^{2t} \begin{bmatrix} 1 \\ -2 \end{bmatrix} + c_2 e^{3t} \begin{bmatrix} 1 \\ -1 \end{bmatrix}.
  \]
  Applying initial conditions:
  \[
  c_1 \begin{bmatrix} 1 \\ -2 \end{bmatrix} + c_2 \begin{bmatrix} 1 \\ -1 \end{bmatrix} = \begin{bmatrix} 2 \\ 1 \end{bmatrix}.
  \]
  This gives the system:
  \[
  \begin{cases}
    c_1 + c_2 = 2 \\
    -2c_1 - c_2 = 1
  \end{cases}
  \]
  Solving this system yields \( c_1 = -3 \), \( c_2 = 5 \). Thus, the solution is:
  \[
  \vec{X}(t) = -3 e^{2t} \begin{bmatrix} 1 \\ -2 \end{bmatrix} + 5 e^{3t} \begin{bmatrix} 1 \\ -1 \end{bmatrix}.
  \]

  \item Describe the behavior of the solution in the phase plane.

  Since both eigenvalues are real and positive (\( \lambda = 2, 3 \)), and the eigenvectors are linearly independent, the system represents an **unstable node**. The solutions will exponentially diverge from the origin along the directions defined by the eigenvectors.
\end{enumerate}

\section*{Question 3: AC RL Circuit and Complex Impedance}

An alternating voltage source \( V(t) = V_0 e^{i\omega t} \) is applied across a resistor \( R \) and an inductor \( L \) connected in series.

\begin{enumerate}[a)]
  \item Using the concept of complex impedance, derive the total impedance \( Z \) of the RL circuit in terms of \( R \), \( L \), and the angular frequency \( \omega \).

  The impedance of the resistor is \( Z_R = R \), and the impedance of the inductor is \( Z_L = i\omega L \). The total impedance is the sum of these two:
  \[
  Z = Z_R + Z_L = R + i\omega L.
  \]

  \item Using Ohm's law in the complex form \( V(t) = I(t) Z \), find the expression for the current \( I(t) \) in the circuit.

  From Ohm's law, we have:
  \[
  I(t) = \frac{V(t)}{Z} = \frac{V_0 e^{i\omega t}}{R + i\omega L}.
  \]
  Multiplying numerator and denominator by the complex conjugate of \( Z \), we get:
  \[
  I(t) = \frac{V_0 e^{i\omega t}(R - i\omega L)}{R^2 + (\omega L)^2}.
  \]
  Thus, the current is:
  \[
  I(t) = \frac{V_0}{\sqrt{R^2 + (\omega L)^2}} e^{i(\omega t - \theta)},
  \]
  where \( \theta = \tan^{-1}\left( \frac{\omega L}{R} \right) \) is the phase shift.

  \item Find the phase shift between the voltage and the current in the circuit.

  The phase shift is given by \( \theta = \tan^{-1}\left( \frac{\omega L}{R} \right) \). This phase shift represents the time delay between the voltage and current waveforms, which depends on the relative contributions of the resistor and inductor to the total impedance.
\end{enumerate}

\section*{Question 4: Fourier Analysis}

Let \( f(x) \) be a periodic function with period \( T \), defined as:
\[
f(x) = \begin{cases}
  1, & 0 \leq x < \frac{T}{2}, \\
  -1, & \frac{T}{2} \leq x < T.
\end{cases}
\]

\begin{enumerate}[a)]
  \item Compute the Fourier coefficients \( a_0 \), \( a_n \), and \( b_n \) for this function.
  
  The Fourier coefficients are computed using the formulas:
  \[
  a_0 = \frac{2}{T} \int_0^T f(x) \, dx = \frac{2}{T} \left( \int_0^{T/2} 1 \, dx + \int_{T/2}^T (-1) \, dx \right) = 0.
  \]
  For \( n \geq 1 \),
  \[
  a_n = \frac{2}{T} \int_0^T f(x) \cos\left( \frac{2\pi n x}{T} \right) \, dx = 0,
  \]
  and
  \[
  b_n = \frac{2}{T} \int_0^T f(x) \sin\left( \frac{2\pi n x}{T} \right) \, dx = \frac{4}{n\pi} \left( 1 - (-1)^n \right).
  \]

  \item Find the Fourier series representation of \( f(x) \).
  
  The Fourier series is:
  \[
  f(x) = \sum_{n=1}^{\infty} \frac{4}{n\pi} \sin\left( \frac{2\pi n x}{T} \right).
  \]

  \item Discuss the convergence of the Fourier series for this function.
  
  The Fourier series for this piecewise function converges to \( f(x) \) at all points except where \( f(x) \) is discontinuous (at \( x = T/2 \)). At these points, the series converges to the average of the left-hand and right-hand limits of the function, according to the Dirichlet conditions.

\end{enumerate}

\section*{Question 5: Vector Calculus}

Let \( \mathbf{F} = x^2 \hat{i} + 2xy \hat{j} + yz \hat{k} \) be a vector field.

\begin{enumerate}[a)]
  \item Compute the divergence of \( \mathbf{F} \), \( \nabla \cdot \mathbf{F} \).
  
  \[
  \nabla \cdot \mathbf{F} = \frac{\partial}{\partial x} (x^2) + \frac{\partial}{\partial y} (2xy) + \frac{\partial}{\partial z} (yz) = 2x + 2x + y = 2x + y.
  \]

  \item Compute the curl of \( \mathbf{F} \), \( \nabla \times \mathbf{F} \).
  
  \[
  \nabla \times \mathbf{F} = \begin{vmatrix} \hat{i} & \hat{j} & \hat{k} \\ \frac{\partial}{\partial x} & \frac{\partial}{\partial y} & \frac{\partial}{\partial z} \\ x^2 & 2xy & yz \end{vmatrix}
  = \hat{i} \left( \frac{\partial yz}{\partial y} - \frac{\partial (2xy)}{\partial z} \right) - \hat{j} \left( \frac{\partial yz}{\partial x} - \frac{\partial (x^2)}{\partial z} \right) + \hat{k} \left( \frac{\partial (2xy)}{\partial x} - \frac{\partial (x^2)}{\partial y} \right).
  \]
  Simplifying, we get:
  \[
  \nabla \times \mathbf{F} = \hat{i}(z - 0) - \hat{j}(0 - 0) + \hat{k}(2y - 0) = z \hat{i} + 2y \hat{k}.
  \]

  \item Evaluate the line integral of \( \mathbf{F} \) along the curve \( C \) parameterized by \( \mathbf{r}(t) = \langle t, t^2, t^3 \rangle \) for \( t \) from 0 to 1.
  
  The line integral is given by:
  \[
  \int_C \mathbf{F} \cdot d\mathbf{r} = \int_0^1 \mathbf{F}(\mathbf{r}(t)) \cdot \frac{d\mathbf{r}(t)}{dt} dt.
  \]
  First, compute \( \frac{d\mathbf{r}(t)}{dt} = \langle 1, 2t, 3t^2 \rangle \). Next, substitute \( \mathbf{r}(t) = \langle t, t^2, t^3 \rangle \) into \( \mathbf{F} \):
  \[
  \mathbf{F}(t, t^2, t^3) = \langle t^2, 2t^3, t^5 \rangle.
  \]
  Then, the dot product is:
  \[
  \mathbf{F} \cdot \frac{d\mathbf{r}}{dt} = t^2(1) + 2t^3(2t) + t^5(3t^2) = t^2 + 4t^4 + 3t^7.
  \]
  Finally, integrate with respect to \( t \):
  \[
  \int_0^1 (t^2 + 4t^4 + 3t^7) \, dt = \left[ \frac{t^3}{3} + \frac{4t^5}{5} + \frac{3t^8}{8} \right]_0^1 = \frac{1}{3} + \frac{4}{5} + \frac{3}{8}.
  \]
  The final answer is:
  \[
  \int_C \mathbf{F} \cdot d\mathbf{r} = \frac{1}{3} + \frac{4}{5} + \frac{3}{8} = \frac{40 + 24 + 15}{120} = \frac{79}{120}.
  \]
\end{enumerate}

\end{document}


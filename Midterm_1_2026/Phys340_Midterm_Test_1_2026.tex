\documentclass[12pt]{article}

\usepackage[margin=1in]{geometry}
\usepackage{amsmath, amssymb}
\usepackage{physics}
\usepackage{enumitem}

\setlength{\parindent}{0pt}
\setlength{\parskip}{0.5em}

\begin{document}

\begin{center}
    {\Large \textbf{Phys 340 -- Second Midterm Test}} \\
    \vspace{0.2cm}
    1 hour and 15 minutes \\
    Professor: Dr. Edward J. Brash
\end{center}

\vspace{0.5cm}

\textbf{Rules and Regulations:}

\begin{enumerate}
    \item Calculators, with memory cleared, are permitted.
    \item You may bring as many pencils, pens, and erasers with you as you like.
    \item You may use your notes, as needed.
    \item The test consists of three (3) questions where you should present full solutions. 
    The full solution questions are worth 10 points each (30 points total).
    \item You should complete your solutions to the full solution questions on the exam paper itself.
    \item Your solutions should contain a combination of diagrams, equations, and English word sentences explaining your reasoning.
    \item If you cannot complete one part of a question, you may use the given result of that part in subsequent sections.
\end{enumerate}

\vspace{0.5cm}

STUDENT NAME: \rule{8cm}{0.4pt}

\vspace{0.3cm}

STUDENT ID NUMBER: \rule{6cm}{0.4pt}

\vspace{0.3cm}

SIGNATURE: \rule{8cm}{0.4pt}

\vspace{1cm}

% ============================================================
\newpage
\section*{1. Vector Algebra (10 points)}

Consider the three vectors

\[
\vec{A} = 2\hat{i} - \hat{j} + 3\hat{k}
\]

\[
\vec{B} = -\hat{i} + 4\hat{j} + 2\hat{k}
\]

\[
\vec{C} = 3\hat{i} + \hat{j} - \hat{k}
\]

\begin{enumerate}[label=(\alph*)]

\item Compute the scalar triple product
\[
\vec{A} \cdot (\vec{B} \times \vec{C})
\]

\vspace{4cm}

\item Compute the quantity
\[
\vec{B} \cdot (\vec{C} \times \vec{A})
\]

\vspace{4cm}

\item By comparing your results from parts (a) and (b), show explicitly that
\[
\vec{A} \cdot (\vec{B} \times \vec{C})
=
\vec{B} \cdot (\vec{C} \times \vec{A})
\]
Explain briefly why this must be true in general.

\vspace{3cm}

\end{enumerate}

% ============================================================
\newpage
Blank page to show your work!
\vfill
\newpage
Blank page to show your work!
\vfill
\newpage
\section*{2. Eigenvalues and Eigenvectors (10 points)}

Consider the following system of coupled first-order differential equations:

\[
\frac{dx}{dt} = 3x + 4y
\]
\[
\frac{dy}{dt} = 4x + 3y
\]

\begin{enumerate}[label=(\alph*)]

\item Write the system in matrix form
\[
\frac{d}{dt}
\begin{pmatrix}
x \\
y
\end{pmatrix}
=
\mathbf{A}
\begin{pmatrix}
x \\
y
\end{pmatrix}
\]
Find the eigenvalues of the transformation matrix $\mathbf{A}$.

\vspace{5cm}

\item Find the normalized eigenvectors corresponding to each eigenvalue.

\vspace{5cm}

\item Given the initial conditions
\[
x(0) = 1, \qquad y(0) = 0
\]
write down the complete time-dependent solution for $x(t)$ and $y(t)$.

\vspace{4cm}

\end{enumerate}

% ============================================================
\newpage
Blank page to show your work!
\vfill
\newpage
Blank page to show your work!
\vfill
\newpage
\section*{3. Moment of Inertia (10 points)}

Consider a thin uniform rod of length $L$ and total mass $M$.  
The rod lies along the $x$-axis, extending from $x=0$ to $x=L$.

\begin{enumerate}[label=(\alph*)]

\item Write an expression for the linear mass density $\lambda$ of the rod.

\vspace{3cm}

\item Calculate the moment of inertia of the rod about the $y$-axis (which passes through the origin and is perpendicular to the rod).

That is, compute
\[
I_y = \int x^2 \, dm
\]
and express your answer in terms of $M$ and $L$.

\vspace{6cm}

\end{enumerate}
\newpage
Blank page to show your work!
\vfill
\newpage
Blank page to show your work!
\vfill
\newpage

\end{document}
